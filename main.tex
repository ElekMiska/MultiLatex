\documentclass[a4paper]{article}

%% Language and font encodings
\usepackage[english]{babel}
\usepackage[utf8x]{inputenc}
\usepackage[T1]{fontenc}

%% Sets page size and margins
\usepackage[a4paper,top=3cm,bottom=2cm,left=3cm,right=3cm,marginparwidth=1.75cm]{geometry}

%% Useful packages
\usepackage{amsmath} 
\usepackage{graphicx}
\usepackage[colorinlistoftodos]{todonotes}
\usepackage[colorlinks=true, allcolors=blue]{hyperref}
\usepackage{commath}
\usepackage{gensymb}
\usepackage{enumitem}
\title{\vspace{-3cm}Homework 4}

\author{\vspace{-1cm}1 hour 45 \\ Alexis Miguel}

\begin{document}
\maketitle
\section{} 
Let P=(1,2,3),      Q=(3,5,7),      R=(2,5,3)

\subparagraph{(a)}
A unit vector perpendicular to a plane containing P, Q, R. 
\subparagraph{}
$\vec{PQ} = <2,3,5>,$
$\vec{PR} = <1,3,0>$

\[
\vec{PQ}X \vec{PR}=
  \begin{bmatrix}
    \hat{i} & \hat{j} & \hat{k} \\
    1 & 3 & 4 \\
    1 & 3 & 0\\
  \end{bmatrix}
  = \begin{bmatrix}
    3 & 4 \\
    3 & 0 \\
  \end{bmatrix} \hat{i}-
  \begin{bmatrix}
    2 & 4 \\
    1 & 0 \\
  \end{bmatrix} \hat{j}+
  \begin{bmatrix}
    2 & 3 \\
    1 & 3 \\
  \end{bmatrix} \hat{k} 
\]
\begin{equation*}
\begin{split}
& =(0-12)\hat{i}-(0-4)\hat{j}+(6-3)\hat{k} \\
& =-12\hat{i}+4\hat{j}+3\hat{k}\\
\end{split}
\end{equation*}
\begin{equation*}
\begin{split}
&\abs{\vec{PQ}X \vec{PR}}= \sqrt[]{(-12)^2+4^2+3^2}=\sqrt[]{13}\\
&\frac{\vec{PQ}X \vec{PR}}{\abs{\vec{PQ}X \vec{PR}}} = <\frac{-12}{13}, \frac{4}{13},\frac{3}{13} >\\
\end{split}
\end{equation*}
\subparagraph{(b)}
The angle between PQ and PR 
\begin{equation*}
\begin{split}
\cos{\theta} &=  \frac{\vec{PQ}\cdot \vec{PR}}{\abs{\vec{PQ}}\abs{\vec{PR}}}=\frac{2+9+0}{\sqrt[]{2^2+3^2+4^2} \sqrt[]{1^2+3^2+0^2}} =\frac{11}{\sqrt[]{290}}\\
\theta&= \arccos{\frac{11}{\sqrt[]{290}}} \approx0.87 radians \approx49.8 ^{\circ}
\end{split}
\end{equation*}

\subparagraph{(c)}
The area of the triangle PQR
\begin{equation*}
\begin{split}
\abs{\vec{PQ}X \vec{PR}} &=\abs{<-12,4,3>} \\
&=\sqrt[]{(-12)^2+4^2+3^2}\\
&=13\\
Area&= \frac{13}{2}=6.5\\ 
\end{split}
\end{equation*}


\section{}
\subparagraph{}
If $\vec{v}X \vec{w} =2\hat{i}-3\hat{j}+5\hat{k}$ and $\vec{v}\cdot \vec{w}=3$. Find $\tan{\theta}$ where $\theta$ is the angle between $\vec{v}$ and $\vec{w}$. Give answer to three places.
\begin{equation*}
\begin{split}
\abs{\vec{v}X \vec{w}}
&= \abs{\vec{v}} \abs{\vec{w}} \sin{\theta}\\
&=\sqrt[]{2^2+(-3)^2+5^2}\\
&=\sqrt[]{38}\\
\end{split}
\end{equation*}
\begin{equation*}
\begin{split}
\vec{v}\cdot\vec{w}=\abs{\vec{v}}\abs{\vec{w}}\cos{\theta}=3
\end{split}
\end{equation*}
\begin{equation*}
\begin{split}
\tan{\theta}&=\frac{\sqrt[]{38}}{3}\\ 
&= \arctan{\frac{\sqrt[]{38}}{3}} \approx 1.118 radians \approx64.049^{\circ}
\end{split}
\end{equation*}

\section{}
Give an example of $\vec{u}$ whose cross product with $\vec{v}=\hat{i}+\hat{j}$ is parallel to $\hat{k}$\\
$\vec{v}$= <1,1,0> \\
$\vec{u}$= <1,1,1>\\
proof: 
\begin{equation*}
\begin{split}
\vec{u}X \vec{v} &=
  \begin{bmatrix}
    \hat{i} & \hat{j} & \hat{k} \\
    1 & 1 & 1 \\
    1 & 1 & 0\\
  \end{bmatrix}
  = \begin{bmatrix}
    1 & 1 \\
    1 & 0 \\
  \end{bmatrix} \hat{i}-
  \begin{bmatrix}
    1 & 1 \\
    1 & 0 \\
  \end{bmatrix} \hat{j}+
  \begin{bmatrix}
    1 & 1 \\
    1 & 1 \\
  \end{bmatrix} \hat{k} \\
&=-1 \hat{i}+1\hat{j}+0\hat{k}  \\
&=-\hat{i}+\hat{j}
\end{split}
\end{equation*}

\section{}
Determine whether the following is a scalar, a vector, or nonsense. 
\begin{enumerate}[label=(\alph*)] 
    \item
    ($\vec{a}\cdot\vec{b}$) $\cdot$ $\vec{c}$ \\ Nonsense
    \item
    ($\vec{a}\ X \vec{b}$) $\cdot$ $\vec{c}$ \\ Scalar
    \item
    ($\vec{a}\cdot\vec{b}$) X $\vec{c}$ \\Nonsense
    \item
    ($\vec{a} X \vec{b}$) X $\vec{c}$ \\ Vector
\end{enumerate}
\section{}
I got stuck on problem 3. To get unstuck, I looked over the class notes and then I read the section in the textbook that corresponded to the problem topic. 

\end{document}